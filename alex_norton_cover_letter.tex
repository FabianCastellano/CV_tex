% start of file `template.tex'.
%% Copyright 2006-2013 Xavier Danaux (xdanaux@gmail.com).
%
% This work may be distributed and/or modified under the
% conditions of the LaTeX Project Public License version 1.3c,
% available at http://www.latex-project.org/lppl/.

\documentclass{my_cv}

% adjust the page margins
%\setlength{\hintscolumnwidth}{3cm}                % if you want to change the width of the column with the dates
%\setlength{\makecvtitlenamewidth}{10cm}           % for the 'classic' style, if you want to force the width allocated to your name and avoid line breaks. be careful though, the length is normally calculated to avoid any overlap with your personal info; use this at your own typographical risks...

% personal data
%\extrainfo{additional information}                 % optional, remove / comment the line if not wanted
%photo[64pt][0.4pt]{picture}                       % optional, remove / comment the line if not wanted; '64pt' is the height the picture must be resized to, 0.4pt is the thickness of the frame around it (put it to 0pt for no frame) and 'picture' is the name of the picture file

\quote{Seeking a position as a Software Developer Trainee}

% to show numerical labels in the bibliography (default is to show no labels); only useful if you make citations in your resume
%\makeatletter
%\renewcommand*{\bibliographyitemlabel}{\@biblabel{\arabic{enumiv}}}
%\makeatother
%\renewcommand*{\bibliographyitemlabel}{[\arabic{enumiv}]}% CONSIDER REPLACING THE ABOVE BY THIS

% bibliography with mutiple entries
%\usepackage{multibib}
%\newcites{book,misc}{{Books},{Others}}
%----------------------------------------------------------------------------------
%            content
%----------------------------------------------------------------------------------
\begin{document}
%\begin{CJK*}{UTF8}{gbsn}                          % to typeset your resume in Chinese using CJK
%-----       resume       ---------------------------------------------------------
%-----       letter       ---------------------------------------------------------
% recipient dat
\newcommand{\companyName}{The University of Texas} % use a space at the end to make things easier below
\newcommand{\jobTitle}{engineering scientist associate }
\recipient{\companyName \ Recruitment team}{\jobTitle position }
\date{\today}
\opening{Dear Sir or Madam,}
\closing{Sincerely,}
\enclosure[Attached]{curriculum vit\ae{}}          % use an optional argument to use a string other than "Enclosure", or redefine \enclname
\makelettertitle

I am extremely interested in the position of \jobTitle with
\companyName.

I would treasure the opportunity to work as a engineering scientist associate.  As a developer at IBM, I learned how to work under an agile paradigm with weekly “SCRUMS”, that helped manage ever changing requirements for software projects.  I worked on an app called Watson Workspace, contributing largely to the notifications and paging system upon opening the mobile application.  I also learned how to use the new iOS programming language, Swift, while on the job and with no loss of efficiency. 

I have experience as an entrepreneur running my own website.  I learned how to use test driven development with Ruby on Rails with Postgres.  Writing all the software on my end taught me how to manage code and the importance of testing. 

Thank you for the opportunity, I would love to work at the University of Austin, a great place with excellent people to work with.  

The programming and planning perspective required for a \jobTitle  position
align with my abilities and experiences.  I am enthusiastic about expanding my
knowledge of programming, and embarking on new challenges.  In addition, I
believe I have a lot to offer \companyName \ in terms of programming expertise and
I am excited about the prospect of working for you.

Thank you for your consideration.  I look forward to meeting you to discuss how
I can support the mission of \companyName.

\makeletterclosing
%\clearpage\end{CJK*}% if you are typesetting your resume in Chinese using CJK; the \clearpage is required for fancyhdr to work correctly with CJK, though it kills the page numbering by making \lastpage undefined
\end{document}
